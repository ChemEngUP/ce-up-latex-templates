\section{Theory}
Summary of the relevant theory, including ample references. You can refer to citations like~\textcite{bruckmanmandersloot} for an inline reference or \parencite{bruckmanmandersloot} to get the citation in parentheses. You can also cite multiple authors~\parencite{mandersloot,bruckmanmandersloot}

The height $h$ \nomenclature{h}{Height\nomunit{m}} was measured as a function of time $t$ \nomenclature{t}{Time\nomunit{s}}.  The relationship between $h$ and $t$ was found to be 
\begin{equation}
  \label{eq:commaexample}
  h(t) = h_0 - \lambda t
\end{equation}
where $h_0$ is the initial height and $\lambda$ is an experimentally
determined constant.  It was also determined that the velocity
$\mathbf{v}$ was a function of the position $\mathbf{x}$ given by
$f(\mathbf{v}) = \beta \mathbf{x} + \sin(\alpha)$ and that in general
$h_{\mathrm{max}} \leq 5$ m.

Dynamic analysis showed that the system could be represented by
Equation~\ref{eq:matexample}
\begin{equation}
  \label{eq:matexample}
  G(s) = \frac{1}{s+1}\left [ 
    \begin{array}{cc} 
      3 & \log{b} \\ 
      2 & \num{1.1} 
    \end{array} \right ]
\end{equation}

Therefore, the model was revised to
\begin{equation}
  \frac{\partial f}{\partial t} = 5t + g(t) \qquad \frac{\mathrm{d} g}{\mathrm{d} t} = t + f(t) \qquad A = \int_a^b f(t) \mathrm{d} t
\end{equation}

For chemical symbols, use the mhchem package to get nicely typeset chemical
names. You can use it in the text like \ce{H2O} or in equations like below:
\begin{align}
  \label{eq:chemexample}
  \ce{H2 + O2 &-> H2O} \\
  \ce{H2O + H+ &-> H3O+}
\end{align}

